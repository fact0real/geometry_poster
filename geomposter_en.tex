\documentclass[17pt, a0paper, landscape]{tikzposter} % See Section 3
\tikzposterlatexaffectionproofoff % disable watermark

\usepackage[T2A]{fontenc}
\usepackage[utf8]{inputenc}
\usepackage{polyglossia}
% Swapped languages for English output
\setmainlanguage{english}
\setotherlanguage{russian}

\usepackage{amsmath,amssymb}
\usepackage[shortlabels]{enumitem}
\usepackage{graphicx}
\usepackage{comment}
\usepackage{parskip}
\usepackage[unicode]{hyperref}
\usepackage{pgfplots}
\pgfplotsset{compat=1.15}
\usepackage{mathrsfs}
\usetikzlibrary{arrows}

% \bulurl command to make blue hyperlink
\usepackage{xcolor}
\usepackage[normalem]{ulem}
\useunder{\uline}{\ulined}{}%
\DeclareUrlCommand{\bulurl}{\def\UrlFont{\ttfamily\color{blue}\ulined}}

% changing font
\usepackage{PTSans}

\setmainfont{PT Sans}
\setromanfont{PT Sans} 
\setsansfont{PT Sans} 
\setmonofont{Consolas} 

\newfontfamily{\cyrillicfont}{PT Sans} 
\newfontfamily{\cyrillicfontrm}{PT Sans}
\newfontfamily{\cyrillicfonttt}{PT Sans}
\newfontfamily{\cyrillicfontsf}{PT Sans}

\usepackage{unicode-math}
\setmathfont{Fira Math}

% define tikzposter colors (mainly 1st one)
\definecolorpalette{PurpleGrayBlue}{
    \definecolor{colorOne}{HTML}{D40279}
    \definecolor{colorTwo}{HTML}{7F8897}
    \definecolor{colorThree}{HTML}{006C9E}
}

% custom title configuration
\settitle{\centering \@title}
% Translated Title
\title{\textcolor[HTML]{D40279}{\textbf{\Huge{Math Exam Prep: Plane Geometry}}}}

% the following bit of code adds logo blocks on the left and right to the title
\makeatletter
\newcommand\insertlogoi[2][]{\def\@insertlogoi{\includegraphics[#1]{#2}}}
\newcommand\insertlogoii[2][]{\def\@insertlogoii{\includegraphics[#1]{#2}}}
\newlength\LogoSep
\setlength\LogoSep{0pt}

\insertlogoi[width=10cm]{figures/logo.pdf}
\insertlogoii[width=5cm]{example-image-b} % I inserted my hyperlink instead of the logo on the right so this line is not used

\renewcommand\maketitle[1][]{  % #1 keys
    \normalsize
    \setkeys{title}{#1}
    % Title dummy to get title height
    \node[transparent,inner sep=\TP@titleinnersep, line width=\TP@titlelinewidth, anchor=north, minimum width=\TP@visibletextwidth-2\TP@titleinnersep]
        (TP@title) at ($(0, 0.5\textheight-\TP@titletotopverticalspace)$) {\parbox{\TP@titlewidth-2\TP@titleinnersep}{\TP@maketitle}};
    \draw let \p1 = ($(TP@title.north)-(TP@title.south)$) in node {
        \setlength{\TP@titleheight}{\y1}
        \setlength{\titleheight}{\y1}
        \global\TP@titleheight=\TP@titleheight
        \global\titleheight=\titleheight
    };

    % Compute title position
    \setlength{\titleposleft}{-0.5\titlewidth}
    \setlength{\titleposright}{\titleposleft+\titlewidth}
    \setlength{\titlepostop}{0.5\textheight-\TP@titletotopverticalspace}
    \setlength{\titleposbottom}{\titlepostop-\titleheight}

    % Title style (background)
    \TP@titlestyle

    % Title node
    \node[inner sep=\TP@titleinnersep, line width=\TP@titlelinewidth, anchor=north, minimum width=\TP@visibletextwidth-2\TP@titleinnersep]
        at (0,0.5\textheight-\TP@titletotopverticalspace)
        (title)
        {\parbox{\TP@titlewidth-2\TP@titleinnersep}{\TP@maketitle}};

    \node[inner sep=400pt,anchor=west] 
      at ([xshift=-\LogoSep]title.west)
      {\@insertlogoi};

    \node[inner sep=400pt,anchor=east] 
      at ([xshift=\LogoSep]title.east)
      {\Huge\bulurl{https://plucik.ru/}}; %\@insertlogoii if you want to have 2nd logo on the right

    % Settings for blocks
    \normalsize
    \setlength{\TP@blocktop}{\titleposbottom-\TP@titletoblockverticalspace}
}
\makeatother
\usetheme{Simple} % See Section 5
\colorlet{titlebgcolor}{white}
\usetitlestyle{Default}
\useblockstyle{TornOut}

% parallel sign
\newcommand{\parallelsum}{\mathbin{\|}}

\begin{document}

% define colors for geogebra imported tikz images
\definecolor{atfczz}{rgb}{0,0.6,0}
\definecolor{fruycc}{rgb}{0.8313725490196079,0.00784313725490196,0.4745098039215686}
\definecolor{qqwuqq}{rgb}{0,0.6,0}
\definecolor{ududff}{rgb}{0.8313725490196079,0.00784313725490196,0.4745098039215686}
\definecolor{qqqqff}{rgb}{0.8313725490196079,0.00784313725490196,0.4745098039215686}
\definecolor{atfcqq}{rgb}{0,0.6,0}
\definecolor{xdxdff}{rgb}{0.8313725490196079,0.00784313725490196,0.4745098039215686}
\definecolor{duqsxz}{rgb}{0.8313725490196079,0.00784313725490196,0.4745098039215686}

\maketitle

% main body of the document, pretty straightforward
\begin{columns}
\column{0.166}
\block{Intercept Theorem (Thales)}
{
$$l_1 \ \parallelsum \ l_2 \ \parallelsum \ l_3 \iff \frac {a} {a'} = \frac {b} {b'} = \frac {c} {c'}$$
\begin{center}
\input{figures/thales_plain.tex}
\end{center}
}
\block{Centroid of a Triangle}
{
$$\frac {CO} {OM} = \frac {AO} {OD} = \frac {BO} {OE} = \frac {2} {1}$$
\begin{center}
\input{figures/median_plain.tex}
\end{center}
}
\block{Trapezoid Property}
{
\begin{center}
$F, M$ - midpoints of bases of trapezoid $ABCD$\\
$\Rightarrow$ points $E, F, M$ are collinear \\
\input{figures/trapezoid_plain.tex}
\end{center}
}
\block{Pythagorean Theorem}
{
$$\Delta ABC \text{ - right-angled } \iff a^2 + b^2 = c^2$$
\begin{center}
\input{figures/pythagoras_plain.tex}
\end{center}
}
\block{Altitude of a Right Triangle}
{
$$h^2 = x \cdot y$$
\begin{center}
\input{figures/h_right_plain.tex}
\end{center}
}

\column{0.166}
\block{Law of Sines}
{
$$\frac {a} {\sin{\alpha}} = \frac {b} {\sin{\beta}} = \frac {c} {\sin{\gamma}} = 2R$$
\begin{center}
\input{figures/sin_plain.tex}
\end{center}
}
\block{Law of Cosines}
{
$$c^2 = a^2 + b^2 - 2ab\cdot\cos{\gamma}$$
\begin{center}
\input{figures/cos_plain.tex}
\end{center}
}
\block{Menelaus' Theorem}
{
$$\frac {AN} {NB} \cdot \frac {BM} {MC} \cdot \frac {CK} {KA} = 1$$
\begin{center}
\input{figures/menelai_plain.tex}
\end{center}
}
\block{Ceva's Theorem}
{
$$\frac {AM} {MB} \cdot \frac {BN} {NC} \cdot \frac {CK} {KA} = 1$$
\begin{center}
\input{figures/cheva_plain.tex}
\end{center}
}
\block{Van Aubel's Theorem}
{
$$\frac {BO} {OK} = \frac {BN} {NC} + \frac {BM} {MA}$$
\begin{center}
\input{figures/van-obel_plain.tex}
\end{center}
}

\column{0.166}
\block{Angle Bisector Theorem}
{
$$\frac {a} {b} = \frac {x} {y}$$
\begin{center}
\input{figures/bisector_plain.tex}
\end{center}
}
\block{Area of a Triangle}
{
$$S_\Delta = \frac 1 2 \cdot ah = \frac 1 2 \cdot ab \cdot \sin{\alpha}$$
\begin{center}
\begin{tikzpicture}[line cap=round,line join=round,>=triangle 45,x=1cm,y=1cm, every node/.style={scale=2}]
\draw[line width=2pt,color=atfczz,fill=atfczz,fill opacity=0.1] (5.569728971410907,0) -- (5.569728971410907,0.5697289714109068) -- (5,0.5697289714109067) -- (5,0) -- cycle; 
\draw [shift={(13,0)},line width=2pt,color=qqwuqq,fill=qqwuqq,fill opacity=0.10000000149011612] (0,0) -- (138.81407483429035:1.2085776573692664) arc (138.81407483429035:180:1.2085776573692664) -- cycle;
\draw [line width=2pt] (0,0)-- (5,7);
\draw [line width=2pt] (5,7)-- (13,0);
\draw [line width=2pt] (5,7)-- (5,0);
\draw [line width=2pt] (0,0)-- (13,0);
\begin{scriptsize}
\draw [fill=ududff] (0,0) circle (2.5pt);
\draw [fill=qqqqff] (13,0) circle (2.5pt);
\draw [fill=qqqqff] (5,7) circle (2.5pt);
\draw [fill=qqqqff] (5,0) circle (2.5pt);
\draw[color=black] (9.128371227823514,4.404974508154897) node {$b$};
\draw[color=black] (4.435061325039529,3.4179694213033307) node {$h$};
\draw[color=black] (6.570215186391899,-0.4) node {$a$};
\draw[color=qqwuqq] (11,0.7) node {$\alpha$};
\end{scriptsize}
\end{tikzpicture}
\end{center}
}
\block{Heron's Formula}
{
$$p = \frac {a + b + c} {2},$$
$$S_\Delta = \sqrt{p(p-a)(p-b)(p-c)}$$
\vspace{0.5cm}
\begin{center}
\input{figures/geron_plain.tex}
\end{center}
}
\block{Area of a Trapezoid}
{
$$S_{ABCD} = \frac {AD + BC} {2}\cdot h$$
\begin{center}
\input{figures/trapezoid_area_plain.tex}
\end{center}
}
\block{Area of a Quadrilateral}
{
$$S_{ABCD} = \frac 1 2 \cdot AC \cdot BD \cdot \sin{\alpha}$$
\begin{center}
\input{figures/4p_area_plain.tex}
\end{center}
}

\column{0.166}
\block{Ratio of areas of triangles with common angle}
{
$$\frac {S_{\Delta ABC}} {S_{\Delta ADE}} = \frac {AB} {AD} \cdot \frac {AC} {AE}$$
\begin{center}
\input{figures/area_frac_plain.tex}
\end{center}
}
\block{Angles subtended by an arc}
{
$$\angle{AEB} = \angle{ADB} = \frac 1 2 \smallsmile{AB}$$
\begin{center}
\input{figures/duga_plain.tex}
\end{center}
}
\block{Angle between tangent and chord}
{
$$\angle{ABC} = \angle{ADB} = \frac 1 2 \smallsmile{AB}$$
\begin{center}
\input{figures/kas_angle_plain.tex}
\end{center}
}
\block{Angle between chords}
{
$$\angle{AKB} = \angle{EKD} = \frac {\smallsmile{AB} \ + \smallsmile{ED}} {2}$$
\begin{center}
\input{figures/chord_angle_plain.tex}
\end{center}
}
\block{Angle between secants}
{
$$\angle{BAC} = \frac {\smallsmile {DE} \ - \smallsmile{BC}} {2}$$
\begin{center}
\input{figures/sec_angle_plain.tex}
\end{center}
}

\column{0.166}
\block{Angles formed by parallel lines}
{
$$k \ \parallelsum \ l \iff \alpha = \beta = \gamma$$
\begin{center}
\input{figures/parallel_plain.tex}
\end{center}
}
\block{Tangents drawn from one point}
{
$$AB = AC$$
\begin{center}
\input{figures/2kas_plain.tex}
\end{center}
}
\block{Tangent-Secant Theorem}
{
$$\Delta ABC \sim \Delta ADB, \quad AB^2 = AC \cdot AD$$
\begin{center}
\input{figures/kas_plain.tex}
\end{center}
}
\block{Intersecting Chords Theorem}
{
$$\Delta AKB \sim \Delta EKD, \quad AK \cdot KD = BK \cdot KE$$
\vspace{0.5cm}
\begin{center}
\input{figures/chord_plain.tex}
\end{center}
}
\block{Secant-Secant Theorem}
{
$$\Delta ABC \sim \Delta ADE, \quad AB \cdot AD = AC \cdot AE$$
\begin{center}
\input{figures/sec_plain.tex}
\end{center}
}

\column{0.166}
\block{Incircle of a Triangle}
{
\begin{center}
The incenter is the intersection point of the angle bisectors \\
\vspace{0.5cm}
\input{figures/cvt_plain.tex}
\end{center}
}
\block{Circumcircle of a Triangle}
{
\begin{center}
The circumcenter is the intersection point of the perpendicular bisectors \\
\vspace{0.5cm}
\input{figures/cot_plain.tex}
\end{center}
}
\block{Circumcircle of a Right Triangle}
{
\begin{center}
$\Delta ABC$ - right-angled $\iff$ $AC$ - diameter,\\
AO = OC = OB = R\\
\vspace{0.5cm}
\input{figures/ort_plain.tex}
\end{center}
}
\block{Cyclic Quadrilateral}
{
\begin{center}
$ABCD$ - cyclic $\iff \alpha + \beta = 180^\circ$\\
\vspace{0.5cm}
\begin{tikzpicture}[line cap=round,line join=round,>=triangle 45,x=1cm,y=1cm, every node/.style={scale=2}]
\draw [shift={(1.2391441753682169,5.551647976431845)},line width=2pt,color=atfczz,fill=atfczz,fill opacity=0.1] (0,0) -- (-77.90850664840158:0.8212560386473431) arc (-77.90850664840158:14.02165447024648:0.8212560386473431) -- cycle;
\draw [shift={(7.927735410246219,1.9157466518178983)},line width=2pt,color=duqsxz,fill=duqsxz,fill opacity=0.1] (0,0) -- (91.11584740696172:0.8212560386473431) arc (91.11584740696172:179.18568628831366:0.8212560386473431) -- cycle;
\draw [line width=2pt] (5,4.5) circle (3.905124837953328cm);
\draw [line width=2pt] (1.2391441753682169,5.551647976431845)-- (7.82488249596053,7.196300963174513);
\draw [line width=2pt] (7.82488249596053,7.196300963174513)-- (7.927735410246219,1.9157466518178983);
\draw [line width=2pt] (7.927735410246219,1.9157466518178983)-- (2,2);
\draw [line width=2pt] (2,2)-- (1.2391441753682169,5.551647976431845);
\begin{scriptsize}
\draw [fill=qqqqff] (2,2) circle (2.5pt);
\draw[color=qqqqff] (1.6514837819185615,1.8165401426270997) node {$A$};
\draw [fill=xdxdff] (7.927735410246219,1.9157466518178983) circle (2.5pt);
\draw[color=xdxdff] (8.256728778467908,1.769611226132966) node {$D$};
\draw [fill=xdxdff] (1.2391441753682169,5.551647976431845) circle (2.5pt);
\draw[color=xdxdff] (0.7481021394064841,5.711640211640212) node {$B$};
\draw [fill=xdxdff] (7.82488249596053,7.196300963174513) circle (2.5pt);
\draw[color=xdxdff] (8.104209799861973,7.506671267540834) node {$C$};
\draw[color=atfczz] (2.5,4.9) node {$\alpha$};
\draw[color=duqsxz] (7.1,2.7) node {$\beta$};
\end{scriptsize}
\end{tikzpicture}
\end{center}
}
\block{Tangential Quadrilateral}
{
\begin{center}
$ABCD$ - tangential $\iff AB + CD = BC + AD$\\
\vspace{0.5cm}
\input{figures/o4_plain.tex}
\end{center}
}

\end{columns}

\end{document}